
\documentclass[10pt,-letter paper]{article}
\usepackage[left=1in, right=0.75in, top=1in, bottom=0.75in]{geometry}
\usepackage{graphicx} % Required for inserting images
\usepackage{siunitx}
\usepackage{setspace}
\usepackage{gensymb}
\usepackage{xcolor}
\usepackage{caption}
\usepackage{circuitikz}
%\usepackage{subcaption}
\doublespacing
\singlespacing
\usepackage[none]{hyphenat}
\usepackage{amssymb}
\usepackage{relsize}
\usepackage[cmex10]{amsmath}
\usepackage{mathtools}
\usepackage{amsmath}
\usepackage{commath}
\usepackage{amsthm}
\interdisplaylinepenalty=2500
%\savesymbol{iint}
\usepackage{txfonts}
%\restoresymbol{TXF}{iint}
\usepackage{wasysym}
\usepackage{amsthm}
\usepackage{mathrsfs}
\usepackage{txfonts}
\let\vec\mathbf{}
\usepackage{stfloats}
\usepackage{float}
\usepackage{cite}
\usepackage{cases}
\usepackage{subfig}
%\usepackage{xtab}
\usepackage{longtable}
\usepackage{multirow}
%\usepackage{algorithm}
\usepackage{amssymb}
%\usepackage{algpseudocode}
\usepackage{enumitem}
\usepackage{mathtools}
%\usepackage{eenrc}
%\usepackage[framemethod=tikz]{mdframed}
\usepackage{listings}
%\usepackage{listings}
\usepackage[latin1]{inputenc}
%%\usepackage{color}{   
%%\usepackage{lscape}
\usepackage{textcomp}
\usepackage{titling}
\usepackage{hyperref}
%\usepackage{fulbigskip}   
\usepackage{tikz}
\usepackage{graphicx}
\lstset{frame=single,breaklines=true}
\let\vec\mathbf{}
\usepackage{enumitem}
\usepackage{graphicx}
\usepackage{siunitx}
\let\vec\mathbf{}
\usepackage{enumitem}
\usepackage{graphicx}
\usepackage{enumitem}
\usepackage{tfrupee}
\usepackage{amsmath}
\usepackage{amssymb}
\usepackage{mwe} % for blindtext and example-image-a in example
\usepackage{wrapfig}
\graphicspath{{figs/}}
\providecommand{\cbrak}[1]{\ensuremath{\left\{#1\right\}}}
\providecommand{\brak}[1]{\ensuremath{\left(#1\right)}}
\newcommand{\sgn}{\mathop{\mathrm{sgn}}}
\providecommand{\abs}[1]{\left\vert#1\right\vert}
\providecommand{\res}[1]{\Res\displaylimits_{#1}} 
\providecommand{\norm}[1]{\left\lVert#1\right\rVert}
%\providecommand{\norm}[1]{\lVert#1\rVert}
\providecommand{\mtx}[1]{\mathbf{#1}}
\providecommand{\mean}[1]{E\left[ #1 \right]}
\providecommand{\fourier}{\overset{\mathcal{F}}{ \rightleftharpoons}}
%\providecommand{\hilbert}{\overset{\mathcal{H}}{ \rightleftharpoons}}
\providecommand{\system}{\overset{\mathcal{H}}{ \longleftrightarrow}}
%\newcommand{\solution}[2]{\textbf{Solution:}{#1}}
%\newcommand{\solution}{\noindent \textbf{Solution: }}
\newcommand{\cosec}{\,\text{cosec}\,}
\providecommand{\dec}[2]{\ensuremath{\overset{#1}{\underset{#2}{\gtrless}}}}
\newcommand{\myvec}[1]{\ensuremath{\begin{pmatrix}#1\end{pmatrix}}}
\newcommand{\myaugvec}[2]{\ensuremath{\begin{amatrix}{#1}#2\end{amatrix}}}
\newcommand{\mydet}[1]{\ensuremath{\begin{vmatrix}#1\end{vmatrix}}}

\begin{document}

\author{Sujal Gupta$^{*}$ FWC22259}
\title{GATE EC2016, 43}
\date{\today}

\maketitle

\bigskip
The functionality implemented by the circuit below is

\begin{figure}[!ht]
\centering

\begin{circuitikz}[scale=1.1]
\tikzstyle{every node}=[font=\large]
\draw  (5,10.75) rectangle (7.75,7);
\draw [](4.75,16) to[short] (11.75,16);
\draw [](4.75,14.75) to[short] (11.75,14.75);
\draw [](4.75,13.25) to[short] (11.75,13.25);
\draw [](4.75,17.5) to[short] (11.75,17.5);
\draw (11.5,17.5) node[ieeestd buffer port, anchor=in](port){} (port.out) to[short] (13.25,17.5);
\draw (port.in) to[short] (11.25,17.5);
\draw (11.5,16) node[ieeestd buffer port, anchor=in](port){} (port.out) to[short] (13.5,16);
\draw (port.in) to[short] (11,16);
\draw (11.5,14.75) node[ieeestd buffer port, anchor=in](port){} (port.out) to[short] (13,14.75);
\draw (port.in) to[short] (11.5,14.75);
\draw (11.5,13.25) node[ieeestd buffer port, anchor=in](port){} (port.out) to[short] (13.25,13.25);
\draw (port.in) to[short] (11.25,13.25);
\draw [](12.25,17.25) to[short] (12.25,16.75);
\draw[] (12.25,16.75) to[short] (9.25,16.75);
\draw [](9.25,16.75) to[short] (9.25,10.25);
\draw[] (9.25,10.25) to[short] (7.75,10.25);
\draw [](12.25,15.75) to[short] (12.25,15.5);
\draw[] (12.25,15.5) to[short] (10,15.5);
\draw [](10,15.5) to[short] (10,9.75);
\draw[] (10,9.75) to[short] (7.75,9.75);
\draw [](12.25,14.5) to[short] (12.25,14);
\draw[] (12.25,14) to[short] (10.75,14);
\draw [](10.75,14) to[short] (10.75,9.25);
\draw [](10.75,9.25) to[short] (10.75,8.75);
\draw[] (10.75,8.75) to[short] (7.75,8.75);
\draw [](12.25,13) to[short] (12.25,7.75);
\draw[] (12.25,7.75) to[short] (7.75,7.75);
\draw [](4,10) to[short] (5,10);
\draw [](4,8) to[short] (5,8);
\draw [](13.25,17.5) to[short] (13.75,17.5);
\draw [](13.75,17.5) to[short] (13.75,13.25);
\draw [](13.25,13.25) to[short] (13.75,13.25);
\draw [](13,14.75) to[short] (13.75,14.75);
\draw [](13.5,16) to[short] (13.75,16);
\draw [](13.75,15.5) to[short] (15.25,15.5);
\node [font=\LARGE] at (4.75,18) {P};
\node [font=\LARGE] at (4.75,16.75) {Q};
\node [font=\LARGE] at (4.75,15.25) {R};
\node [font=\LARGE] at (4.75,13.75) {S};
\node [font=\LARGE] at (3.75,10.5) {$C_1$};
\node [font=\LARGE] at (3.5,8.5) {$C_2$};
\node [font=\LARGE] at (15.25,15.75) {Y};
\node [font=\LARGE] at (8.75,10.75) {$O_0$};
\node [font=\LARGE] at (9,9.25) {$O_1$};
\node [font=\LARGE] at (9,8.25) {$O_2$};
\node [font=\LARGE] at (8.75,6.75) {$O_3$};
\draw [](6.25,7) to[short] (6.25,5.25);
\node [font=\large] at (4.5,6) {Enable=1};
\end{circuitikz}


\label{fig:multiplexer}
\end{figure}

\begin{enumerate}[label=\Alph*.]
\item 2-to-1 multiplexer
\item 4-to-1 multiplexer
\item 7-to-1 multiplexer
\item 6-to-1 multiplexer
\end{enumerate}

\end{document}
